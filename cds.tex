\newpage
\section*{'കുലസ്ത്രീയും' 'ചന്തപ്പെണ്ണും' ഉണ്ടായതെങ്ങനെ ?}

കേരളമാത്യകാസ്ത്രീത്വം ഏറെ പ്രകീർത്തിക്കപ്പെട്ടപ്പോഴും കേരളത്തെ കേന്ദ്രീകരിച്ചുള്ള വിജ്ഞാനശാഖകളിൽ മലയാളിസ്ത്രീകളുടെ ശബ്ദവും സാന്നിദ്ധ്യവും അടുത്തകാലംവരെയും കുറഞ്ഞുതന്നെയാണിരുന്നത്. തന്നെയുമല്ല, കേരളത്തിലെ ലിംഗബന്ധങ്ങളുടെ ചരിത്രത്തെക്കുറിച്ച് നിലവിലുള്ള പ്രബലധാരണകൾ സമകാലിക മലയാളി സമൂഹത്തിന്റെ ലിംഗപരമായ പ്രത്യേകതകളെ, സമസ്യകളെ, തിരിച്ചറിയാനോ വിശദീകരിക്കാനോ നമ്മെ സഹായിക്കുന്നില്ല.

ഈ പ്രശ്നങ്ങൾ ഒരളവുവരെ ഇന്ന് പരിഹരിക്കപ്പെടുന്നുണ്ട്. ചരിത്രമെന്നാൽ രാഷ്ട്രീയ ചരിത്രം എന്ന വ്യവസ്ഥാപിതധാരണ മാറുന്നതനുസരിച്ച് വർത്തമാനകാലപ്രശ്നങ്ങളെക്കുറിച്ചുള്ള പൊതുചർച്ചകളിൽ ചരിത്ര വിജ്ഞാനത്തിനും വീക്ഷണത്തിനും മുന്തിയപ്രാധാന്യമുണ്ടെന്ന് നാം സമ്മതിച്ചു തുടങ്ങിയിരിക്കുന്നു. പുരാരേഖാശേഖരങ്ങളിലേക്ക് പ്രവേശം സിദ്ധിച്ച ഏതാനും പേർ മാത്രം പങ്കുവെയ്ക്കേണ്ട വിജ്ഞാനമല്ല ചരിത്രമെന്നും അത് വരേണ്യവർഗങ്ങളുടെയോ ദേശരാഷ്ട്രത്തിന്റെയോ വ്യവഹാരം മാത്രമായിക്കൂടെന്നുമുള്ള തിരിച്ചറിവ് ഇന്ന് ശക്തമാണ്.

ഈ വെളിച്ചത്തിൽ, കൂടുതൽ തുല്യത, നീതി, ജനാധിപത്യം എന്നിവയിലൂന്നിയ പുതിയസാമൂഹ്യബന്ധങ്ങൾ നിർമ്മിക്കാനും വ്യക്തികൾക്ക് കാലത്തിന്റെ വെല്ലുവിളികളെ നേരിടാനും ഉതകുന്ന വിജ്ഞാനമായി ചരിത്രവിജ്ഞനത്തെ പുനരവതരിപ്പിക്കാനുള്ള ശ്രമമാണ് ഈ പുസ്തകത്തിൽ. ആധുനികകേരളീയസ്ത്രീത്വത്തിന്റെ രൂപീകരണമാണ് ഇതിന്റെ മുഖ്യവിഷയമെങ്കിലും ലിംഗബന്ധങ്ങളുടെ വിശാലചരിത്രത്തിലേക്കും ഇത് വെളിച്ചം വീശുന്നു.

സാമാന്യവായനക്കാർക്കും ചരിത്രപഠനത്തിലേക്കു കടക്കാനുദ്ദേശിക്കുന്ന വിദ്യാർത്ഥികൾക്കും സഹായകമായ ആമുഖപുസ്തകമാണിത്.
\begin{center}
CENTRE FOR
DEVELOPMENT STUDIES
\end{center}
