\chapter*{ആമുഖം}

നിരവധി സുഹൃത്തുക്കളുടെ കൂട്ടായ പരിശ്രമത്തിന്റെ ഫലമാണ് ഈ പുസ്തകം. രണ്ടു വർഷത്തോളം നീണ്ട ഈ ഉദ്യമത്തിൽ സേവാ കേരള, കേരള മഹിളാസമഖ്യ സൊസൈറ്റി എന്നീ സംഘടനകളിലെ സ്ത്രീപ്രവർത്തകരും തിരുവനന്തപുരം, കൊല്ലം ജില്ലകളിലുള്ള ചില ഡിഗ്രി കോളജുകളിലെ വിദ്യാർത്ഥിനീവിദ്യാർത്ഥികളും പങ്കെടുത്തു. കേരളത്തിലെ ആൺ-പെൺ ബന്ധങ്ങളുടെ ചരിത്രത്തെപ്പറ്റി ഇവർ ചോദിക്കുന്ന ചോദ്യങ്ങളെക്കുറിച്ച് ആരാഞ്ഞുകൊണ്ടായിരുന്നു പ്രാരംഭ പ്രവർത്തനം. ഇതിനുശേഷം ആദ്യത്തെ രണ്ട് അദ്ധ്യായങ്ങളെ കരടുരൂപത്തിൽ അവതരിപ്പിച്ചുകൊണ്ട് ഞങ്ങൾ ചർച്ച തുടർന്നു. ഉള്ളടക്കം തീരുമാനിച്ചത് ഈ ചർച്ചകളിലൂടെയായിരുന്നു. പിന്നീട് മുഴുവൻ പുസ്തകത്തിന്റെ കരട് ചർച്ചയ്ക്കുവച്ചു. കേരളത്തിന്റെ പല ഭാഗങ്ങളിൽനിന്ന് തെരഞ്ഞെടുക്കപ്പെട്ട വിദ്യാർത്ഥിനീവിദ്യാർത്ഥികളും മലയാള പൊതുമണ്ഡലത്തിൽ സാന്നിദ്ധ്യമറിയിച്ചിട്ടുള്ളവരുടെ ഒരു സംഘവും സമഖ്യ-സേവാ പ്രവർത്തകരും അദ്ധ്യായങ്ങൾ മുഴുവൻ വായിച്ച് അഭിപ്രായങ്ങളും നിർദ്ദേശങ്ങളുമുന്നയിച്ചു. പുസ്തകത്തിന്റെ അവസാനരൂപം ഇതിലൂടെ തെളിയുകയും ചെയ്തു.


ഈ പ്രക്രിയയിൽ പങ്കുചേർന്ന പലരും എന്റെ അടുത്ത സുഹൃത്തുക്കളാണ്. അവരോട് ഔപചാരികമായി നന്ദി പറയുന്നില്ല - സ്നേഹപൂർവ്വം സ്മരിക്കുന്നുവെങ്കിലും. രഞ്ജിത്, അഞ്ജു, ഫസീല, ചാന്ദ്നി, അന്ന, സിതാര, ദിൽഷദ് എന്നീ വിദ്യാർത്ഥിനീവിദ്യാർത്ഥിസുഹൃത്തുക്കളോട് ഞാൻ പ്രത്യേകം കടപ്പെട്ടിരിക്കുന്നു. വിലപ്പെട്ട വിവരങ്ങൾ തന്നു സഹായിച്ച സുഹൃത്തുക്കളായ - എസ്. സഞ്ജീവ്, ഷംഷാദ്, ബിന്ദു മേനോൻ, ഗീനാകുമാരി, ടി.ടി ശ്രീകുമാർ, ഹസ്സൻകോയ, കെ. സി സന്തോഷ് എന്നിവരോടും ഇതു വായിച്ച് തെറ്റുതിരുത്താനും നിർദ്ദേശങ്ങൾ നൽകാനും ഉത്സാഹിച്ച സുഹൃത്തുക്കളോടും ഗുരുജനങ്ങളോടും പ്രത്യേകം നന്ദി പറയുന്നു. സേവാ കേരള പ്രവർത്തകരായ സോണിയ ജോർജ്, നളിനി നായിക് എന്നിവരും സമഖ്യയുടെ സീമാ ഭാസ്കരനും ഈ പുസ്തകരചനയുടെ ആരംഭംമുതൽക്കേ ഇതിൽ സജീവമായ താൽപര്യം കാട്ടി. അവർക്കും ഹൃദയംനിറഞ്ഞ നന്ദി. കേരളത്തിലെ സ്ത്രീകളുടെയും പെൺകുട്ടികളുടെയും ആശകളിലേക്കും ആശങ്കകളിലേക്കും അല്പമെങ്കിലും ഇറങ്ങിച്ചെല്ലാൻ എന്നെ സഹായിച്ച ആദ്യഘട്ട ചർച്ചകളിൽ പങ്കെടുത്ത സ്ത്രീകളും വിദ്യാർത്ഥിനികളും എന്റെ ഗുരുക്കന്മാർകൂടിയാണ് - ഈ പുസ്തകം അവർക്കു സമർപ്പിക്കുന്നു. ഈ പുസ്തകത്തിന്റെ രൂപകൽപന നിർവ്വഹിച്ച പ്രിയരഞ്ജൻലാലിനെയും സി. ഡി. എസിന്റെ രജിസ്ട്രാർ ശ്രീ. സോമൻനായരെയും സ്നേഹപൂർവ്വം സ്മരിക്കുന്നു. പുസ്തകത്തിന്റെ ആദ്യ കരടുരൂപം തയ്യാറാക്കാൻ ഗ്രീഷ്മയുടെ സഹായം എനിക്കുണ്ടായിരുന്നു. എന്നാൽ തെറ്റുകൾ തിരുത്തി, വായിക്കാവുന്ന നിലവാരമുള്ള കരടുരൂപം തയ്യാറാക്കിത്തന്നത് ശ്രീവാസുദേവ ഭട്ടതിരി, സലിൽ, ഉദയൻ എന്നിവരാണ്. എന്റെ അശ്രദ്ധമായ എഴുത്തിനെ കൂടുതൽ വടിവൊത്തതാക്കാൻ അവർ വളരെ പണിപ്പെട്ടിട്ടുണ്ട്. എന്നാൽ, അവരുടെ പരിശ്രമങ്ങൾക്കുശേഷവും ബാക്കിയായ തെറ്റുകൾക്ക് ഞാൻ മാത്രമാണ് ഉത്തരവാദി. അവസാനമായി, ഈ പുസ്തകമെഴുതാനുള്ള സാഹചര്യമൊരുക്കിത്തന്ന ബാംഗ്ളൂരിലെ സെന്റർ ഫോർ സ്റ്റഡി ഓഫ് കൾച്ചർ ആന്റ് സൊസൈറ്റിയിലെ തേജസ്വിനി നിരഞ്ജന, അശ്വിൻകുമാർ എന്നിവർക്കും എന്റെ നന്ദി രേഖപ്പെടുത്തുന്നു.


ഇത്തരമൊരു പുസ്തകത്തിന്റെ ഉദ്ദേശ്യമെന്താണ്? മലയാളിസ്ത്രീയുടെ ചരിത്രത്തിന്റെ എഴുതാപ്പുറം വെളിവാക്കുന്ന "കൗതുകവിവരങ്ങൾ' എത്തിക്കലല്ല ഇതിന്റെ ലക്ഷ്യമെന്ന് ഉറപ്പായും പറയാം. ബിരുദതലചരിത്രപഠനത്തിൽ വിദ്യാർത്ഥീവിദ്യാർത്ഥിനികൾക്കു സഹായകമായ വസ്തുതകൾ ഈ പുസ്തകത്തിലുണ്ട്; ഒരുപക്ഷേ ആ പഠനത്തിലേക്ക് അവരുടെ ശ്രദ്ധയെ കാര്യമായി തിരിച്ചുവിടാൻ ഇത് ഉതകിയേക്കും. എന്നാൽ, ചരിത്രത്തെ - പഠനവിഷയമെന്ന നിലയ്ക്കും ഭൂതകാലമെന്ന നിലയ്ക്കും - നാം കാണുന്ന രീതിയെ അപ്പാടെ മാറ്റാൻ ഈ പുസ്തകം ഉപകരിക്കുമെന്നാണ് എന്റെ പ്രതീക്ഷ. സമൂഹത്തെ നവീകരിക്കാൻ ഇടയ്ക്കിടെ "മാനസികമായ ഉറയൂരൽ' ആവശ്യമാണ്. അതിനുതകുന്ന പുസ്തകമാകും ഇത് എന്ന് ഞാൻ പ്രത്യാശിക്കുന്നു. "ഉറയൂരലി'നെക്കുറിച്ച് കവി പറഞ്ഞത്:   \footnote{ഈ പുസ്തകത്തിനുവേണ്ടി എ.കെ രാമാനുജന്റെ 'Moulting' എന്ന ഈ കവിത മലയാളത്തിൽ പരിഭാഷപ്പെടുത്തിത്തന്ന ബാലചന്ദ്രൻ ചുള്ളിക്കാടിനെ സ്നേഹപൂർവ്വം സ്മരിക്കുന്നു}


\noindent
ഉറയൂരാൻ ആദ്യം വേണ്ടത്\\
വാലറ്റത്ത് ഏറിവരുന്ന മരവിപ്പിനെ\\
തച്ചുറപ്പിക്കാൻ\\
വേണ്ടത്ര ഉയരത്തിൽ ഒരു\\
മുള്ള് കണ്ടെത്തുകയാണ്.\\

\noindent
പിന്നെ ഉറയൂരൽ തുടങ്ങാം.\\
അയഞ്ഞ വാലിൽനിന്ന് കൂടൊഴിയാം\\
മുഴുവൻ തൊലിയും ജീവനോടെ ഉരിഞ്ഞു\\
പോകുമെങ്കിലും.\\
അങ്ങനെയാണ്\\
ഒന്നോരണ്ടോ ഉണങ്ങിയ പാമ്പുറ\\
തൂങ്ങിക്കിടക്കുന്നത്\\
നീ കാണുക.\\

\noindent
മാത്രമല്ല, നീയിപ്പോൾ മൂത്രമൊഴിച്ച്\\
അശുദ്ധമാക്കിയ\\
വേലിമേൽ, കാരമുള്ളിൽ തൂങ്ങിയാടി\\
പുതുശരീരം രൂപപ്പെടുത്താൻ യത്നിക്കുന്ന\\
മെലിഞ്ഞു വിളറിയ ഒരു വയസ്സൻ പാമ്പ്\\ 
ഒരുനിമിഷം നിന്നെ അസ്വസ്ഥനാക്കാനും മതി.\\

\noindent
പാമ്പുകൾക്കും പരുന്തുകൾക്കും\\
അവയ്ക്കിടയിലുള്ളവയ്ക്കൊക്കെയും\\
ദൈവമായുള്ളോവേ,\\
എന്റെ മകന്റെ പരിവർത്തനമൂഹൂർത്തത്തിൽ\\
ഒരു നാഴികയുടെ തണൽകൊണ്ട്\\
അവനെ നീ മൂടേണമേ.\\
വേണ്ടുന്ന ഉയരത്തിലെ ആ മുള്ളായി\\
അവനെ നീ അനുഗ്രഹിക്കേണമേ.\\

ഈ പുസ്തകം ഞാനെഴുതിയതാണെങ്കിലും അതിന്റെ ഒരു ഭാഗം മാത്രമെ എന്റേതായിട്ടുള്ളൂ. ഇതിന്റെ കലാപരമായ ഉള്ളടക്കം രൂപപ്പെടുത്തിയത് ബി. പ്രിയരഞ്ജൻലാൽ ആണ്. ഇതിലുന്നയിക്കുന്ന വിഷയങ്ങളെ കലയുടെ തനതായ രീതിയിൽ അവതരിപ്പിക്കുന്നതുകൊണ്ടുതന്നെ, എഴുതിയ ഭാഗത്തെയപേക്ഷിച്ച് സ്വതന്ത്രമായ ഒരു നില അതിനുണ്ട്.


സാമാന്യവിദ്യാഭ്യാസം നേടിയ, പൊതുകാര്യങ്ങളിൽ താൽപര്യംകാട്ടുന്ന ഏതൊരു വ്യക്തിക്കും വായിക്കാനും മനസ്സിലാക്കാനും കഴിയുന്ന രീതിയിൽ കാര്യങ്ങൾ പറയാനാണ് ഞാനിവിടെ ശ്രമിച്ചിട്ടുള്ളത്. ഇതിൽ ചർച്ചചെയ്യുന്ന വിഷയങ്ങളെക്കുറിച്ചുള്ള സമഗ്രമായ വിവരണമല്ല, പക്ഷേ, ഈ പുസ്തകത്തിൽ. വായിക്കുന്നവരിൽ ഈ വിഷയങ്ങളെക്കുറിച്ച് പുതുതാൽപര്യം സൃഷ്ടിക്കുകയെന്ന പരിമിതലക്ഷ്യമേ ഈ പുസ്തകത്തിനുള്ളു. ഈ വായനയിലൂടെ വായനക്കാർ കൂടുതൽ വിവരങ്ങളന്വേഷിക്കാൻ പ്രേരിതരാകുമെന്നു ഞാൻ ആശിക്കുന്നു. സ്ത്രീകൾക്കുമാത്രമല്ല, പുരുഷന്മാർക്കും സ്വയംപഠനത്തിന്റെയും സ്വയംവിമർശനത്തിന്റെയും വഴി തെരഞ്ഞെടുക്കാൻ ഈ പുസ്തകം പ്രചോദനമാകുമെന്നാണ് എന്റെ പ്രതീക്ഷ.
\begin{flushright}
{ജെ. ദേവിക}
\end{flushright}
