\chapter{കുടുംബിനിയോ പൗരയോ?}
\label{10}

%%%%%BOX%%%%%%
\captionof{mybox}{'കുറിയേടത്തു താത്രി കേസ്'}\label{ch10box2} % place the caption
\begin{tcolorbox}[%
  breakable, % make the box breakable
  arc=0mm, 
  left=1pt, right = 1pt, 
  boxrule=0mm,
  colback = {blue!10}, % since shadow-gray was not defined
] 

\paragraph{}നമ്പൂതിരിസ്ത്രീകൾ വ്യഭിചാരം ചെയ്യുന്നുവെന്ന് സംശയിക്കപ്പെട്ടാൽ, അവരെ സമുദായവിചാരണയ്ക്കു പാത്രമാക്കിയിരുന്നു: 'സ്മാർത്തവിചാരം' എന്നാണ് ഈ സമ്പ്രദായത്തെ വിളിച്ചിരുന്നത്. വ്യഭിചാരക്കുറ്റം ആരോപിക്കപ്പെട്ട സ്ത്രീയെ കുടുംബക്കാരിൽനിന്നും മാറ്റി ഇല്ലത്തിന്റെ മറ്റൊരുഭാഗത്ത് പാർപ്പിച്ച് അവളെ കടുത്ത ചോദ്യംചെയ്യലിനു വിധേയമാക്കിയിരുന്നു. സ്ത്രീ കുറ്റംസമ്മതിച്ചാൽമാത്രമെ വിചാരണ നടത്തുന്നവർക്ക് അവളെ ശിക്ഷിക്കാൻ കഴിയുമായിരുന്നുള്ളൂ; കുറ്റസമ്മതം നടന്നുകഴിഞ്ഞാൽ സ്ത്രീയെ സമുദായത്തിൽനിന്നുതന്നെ പുറത്താക്കിയിരുന്നു.
ഇത്തരമൊരു കേസ് 20-ാം നൂറ്റാണ്ടിന്റെ ആദ്യവർഷങ്ങളിൽ കൊച്ചീരാജ്യത്ത് വലിയ ഒച്ചപ്പാടിനിടവരുത്തി. 1905-'06 കാലത്തായിരുന്നു അത്. 'കുറിയേടത്തു സാവിത്രി' എന്നായിരുന്നു ദോഷം ആരോപിക്കപ്പെട്ട അന്തർജനത്തിന്റെ പേര്. സ്മാർത്തവിചാരം > കാണുക പുറം 34, 35 < നടക്കുമ്പോൾ കുറ്റംസമ്മതിച്ച സ്ത്രീ തന്റെകൂടെ തെറ്റുചെയ്ത പുരുഷന്മാരുടെ പേരുകൾ വെളിപ്പെടുത്തുകയും അവളുടെയൊപ്പം അവരും സമുദായത്തിൽനിന്ന് പുറത്താവുകയും ചെയ്യുന്ന രീതിയായിരുന്നു പതിവ്. എന്നാൽ ഈ കുറിയേടത്തു താത്രി (സാവിത്രി)യുടെ കാര്യത്തിൽ ഒരു വിശേഷമുണ്ടായി. അവർ ഒന്നല്ല, 64 പുരുഷന്മാരുടെ പേരുകൾ പറഞ്ഞു. അവരോടൊപ്പം 64പേരും പുറത്താകേണ്ടിയിരുന്നു. ഇവരിൽ നമ്പൂതിരി, നായർ, അമ്പലവാസി തുടങ്ങി പല സമുദായങ്ങളിലുമുൾപ്പെട്ട പുരുഷന്മാരുണ്ടായിരുന്നു. കൊച്ചീരാജ്യത്തിലെ വലിയ ഇല്ലങ്ങളിലെ പുരുഷന്മാർ പലരും താത്രിയുടെ ലിസ്റ്റിൽ ഉണ്ടായിരുന്നതുകൊണ്ട് ഈ കേസിൽ പുരുഷന്മാരുടെ വശംകൂടി കണക്കാക്കണമെന്ന അഭിപ്രായമുണ്ടായി. കൊച്ചീരാജാവ് അതിന് അനുമതി കൊടുക്കുകയുംചെയ്തു. അന്തർജനത്തെ കൊച്ചിയിൽ കൊണ്ടുവന്നു പാർപ്പിച്ചു. അവിടെ 'പുരുഷവിചാരം' സംഘടിപ്പിച്ചു - അതായത്, താത്രി പേരെടുത്തുപറഞ്ഞ പുരുഷന്മാരെക്കൊണ്ട് അവരെ ചോദ്യംചെയ്യിപ്പിച്ചു. പക്ഷേ, യാതൊരു ഫലവുമുണ്ടായില്ല. തങ്ങൾ കുറിയേടത്തു താത്രിയുമായി സഹവസിച്ചിട്ടില്ലെന്നു തെളിയിക്കാൻ ഇവർക്കാർക്കും കഴിഞ്ഞില്ലത്രെ! താത്രിയോടൊപ്പം അവരെല്ലാം സമുദായത്തിനു പുറത്തായി.
കുറിയേടത്തു താത്രിക്ക് പിന്നീടെന്തു സംഭവിച്ചു എന്ന് നമുക്കറിയില്ല. അവർ കോയമ്പത്തൂരിലേക്കു തീവണ്ടി കയറിയെന്നാണ് പറഞ്ഞുകേൾക്കുന്നത്.
\end{tcolorbox}
%%%%%%%%%%%
