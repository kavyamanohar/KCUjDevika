\chapter{മാറുന്ന മാതൃത്വം}
\label{chapter6}

\paragraph{}
\label{saraswathy} ഇവരിൽ സരസ്വതിയമ്മ പൊതുവെ 'മാതൃത്വമഹിമ'യെക്കുറിച്ചുള്ള അവകാശവാദത്തെ - അതായത്, സ്ത്രീകൾക്ക് നേടാവുന്ന ഏറ്റവും ഉന്നതമായ, ഉത്കൃഷ്ടമായ സാമൂഹ്യനില അമ്മയുടേതാണ് എന്ന വാദത്തെ - സംശയത്തോടെയാണ് കണ്ടത്. സത്യത്തിൽ അമ്മ എന്ന നില അത്ര മഹത്തരമാണോ എന്ന് തന്റെ കഥകളിലൂടെ അവർ ചോദിച്ചു. മാതൃവാത്സല്യത്തെക്കുറിച്ചുള്ള പുകഴ്ത്തലുകൾ സർവ്വത്ര കേൾക്കാമെങ്കിലും അതത്രവളരെ ഉറപ്പുള്ളതാണോ എന്ന ചോദ്യമാണ് 'വാത്സല്യത്തിന്റെ ഉറവ്' എന്ന കഥയിലൂടെ അവർ ഉന്നയിക്കാൻ ശ്രമിച്ചത്. സ്ത്രീശരീരത്തിന്റെ പ്രത്യേകതകളായ ഗർഭധാരണം, പ്രസവം മുതലായ പ്രക്രിയകൾ സ്ത്രീകൾക്ക് പ്രത്യേകിച്ചൊരു സദാചാരമഹിമയും പ്രദാനം ചെയ്യുന്നില്ലെന്നുമാത്രമല്ല അതവരെ സമൂഹത്തിൽ പിന്നാക്കം വലിക്കുന്നുണ്ടെന്നും അവർ വാദിച്ചു. 'സ്ത്രീസഹജവാസന'കളെക്കുറിച്ചുള്ള പറച്ചിൽ കേൾക്കാൻ സുഖമാണെങ്കിലും സ്ത്രീകളെ ചില വാർപ്പുമാതൃകകളിൽ - സ്ത്രീസ്വഭാവത്തെക്കുറിച്ചുള്ള ചില പരിമിതധാരണകളിൽ - ഒതുക്കുകയല്ലേ ചെയ്യുന്നത്? കുടുംബമെന്ന സ്ഥാപനം നൽകുന്ന സാദ്ധ്യതകൾക്കും ക്ഷമ, ദയ, സഹനശീലം മുതലായ 'സ്ത്രീഗുണങ്ങൾ' നൽകുന്ന സാദ്ധ്യതകൾക്കുമപ്പുറം മറ്റു സാദ്ധ്യതകളില്ലേ? സേവനത്തിനും ത്യാഗത്തിനും ഒന്നുംപോകാതെ 'സ്വന്തംപാടു'നോക്കിനടക്കുന്നവരും പുരുഷന്റേതെന്നു കരുതപ്പെടുന്ന രംഗങ്ങളിൽ അഭിരുചിയും കഴിവുള്ളവരുമായ സ്ത്രീകഥാപാത്രങ്ങൾക്ക് സരസ്വതിയമ്മയുടെ കഥാലോകത്തിൽ വലിയ ഇടമുണ്ട്. 'സഹജവാസന'കളുടെപേരിൽ സ്ത്രീകൾ സമൂഹത്തിന്റെ വിശാല ഇടങ്ങളിൽനിന്ന് പുറത്താക്കാനിടവരരുത് എന്ന മുന്നറിയിപ്പാണ് അവർ നൽകിയത്.

\captionof{mybox}{സരസ്വതി അമ്മ}
\label{saraswathy}
%\label{ch6box2} % place the caption
\begin{tcolorbox}[%
 breakable, % make the box breakable
  arc=0mm, 
  left=1pt, right = 1pt, 
  boxrule=0mm,
  colback = {blue!10}, % since shadow-gray was not defined
] 

ലളിതാംബികയുടെ ശൈലിയിൽനിന്നും തീർത്തും വ്യത്യസ്തമായ കഥകളായിരുന്നു സരസ്വതി അമ്മ എഴുതിയതെങ്കിലും പുതിയ സാമൂഹ്യസാഹചര്യങ്ങൾ സ്ത്രീകൾക്കു നൽകിയ സാദ്ധ്യതകളെപ്പറ്റിയുള്ള വിചിന്തനങ്ങളാണ് അവരുടെ കഥകളിലും. 1919ൽ കുന്നപ്പുഴയിൽ കിഴക്കേവീട്ടിൽ കാർത്ത്യായനിയമ്മയുടെ മകളായി അവർ ജനിച്ചു. 1928 മുതൽ തിരുവനന്തപുരത്തായിരുന്നു സരസ്വതിയമ്മയുടെ കുടുംബം താമസിച്ചത്. തിരുനന്തപുരത്തു സ്കൂൾവിദ്യാഭ്യാസം പൂർത്തിയാക്കിയ അവർ പഠനത്തിൽ മികവു കാട്ടിയിരുന്നു. കുടുംബത്തിന് സാമ്പത്തികശേഷി കുറവായിരുന്നു. വീട്ടിൽനിന്നു പ്രോത്സാഹനം കുറവായിരുന്നു. എങ്കിലും സ്കോളർഷിപ്പിന്റെ പിൻബലത്തിൽ ഗവൺമെന്റ് വിമൻസ് കോളേജിൽ ഇന്റർമീഡിയറ്റിനു ചേർന്നു. വളരെ ബുദ്ധിമുട്ടിയാണ് പിന്നീടു ബിരുദവിദ്യാഭ്യാസം പൂർത്തിയാക്കിയത്. പുരുഷന്മാരായ വിദ്യാർത്ഥികളോട് സമഭാവനയോടെ പെരുമാറിയിരിന്നതുകൊണ്ടും പെൺകുട്ടികൾ കയറിച്ചെല്ലാൻ മടിച്ചിരുന്ന വേദികളിലേക്ക് സധൈര്യം കടന്നുചെന്നിരുന്നതുകൊണ്ടുമാവാം 'വട്ടുസരസ്വതി' എന്ന പരിഹാസപ്പേർ അവർക്കു വീണു എന്ന് സഹപാഠികൾ പറയുന്നു. 1942 ൽ ബി.ഏ പാസായശേഷം കുറച്ചുകാലം അദ്ധ്യാപികയായി പ്രവർത്തിച്ചു. 1945 ൽ ലോക്കൽ ഫണ്ട് ഓഡിറ്റ് ഡിപ്പാർട്ട്മെന്റിൽ ഉദ്യോഗസ്ഥയായി. സ്വന്തം വീട് പണികഴിപ്പിച്ച് ഒറ്റയ്ക്കു താമസിച്ചു. കേരളത്തിൽ അക്കാലത്തു പ്രചരിച്ചുവന്ന പുതിയകുടുംബമൂല്യങ്ങളുടെ സ്ത്രീവിരുദ്ധതയെ വിട്ടുവീഴ്ച്ചയില്ലാതെ തുറന്നുകാട്ടുന്ന കഥകളാണവരുടേത്.
\end{tcolorbox}
