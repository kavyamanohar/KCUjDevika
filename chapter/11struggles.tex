\chapter{സ്ത്രീകളും സമരങ്ങളും}
\label{chapter11}
\section{എന്താണീ 'സമരം'?}
\section{പല തലങ്ങളിലെ സമരങ്ങൾ}
\captionof{mybox}{ജാതിവ്യവസ്ഥയുടെ ഇരകൾ}\label{ch11box1} % place the caption
\begin{tcolorbox}[%
 breakable, % make the box breakable
  arc=0mm, 
  left=1pt, right = 1pt, 
  boxrule=0mm,
  colback = {blue!10}, % since shadow-gray was not defined
] 
\end{tcolorbox}
\paragraph{}
\label{umadevi}
പഴയതും പുതിയതുമായ സദാചാരത്തിനു നിരക്കാത്ത പ്രവൃത്തികളായിരുന്നു കുറിയേടത്തു താത്രിയുടേത്. അവരെയും സമരനായികയായി അംഗീകരിക്കാൻ സമുദായപരിഷ്ക്കർത്താക്കളും വ്യവസ്ഥാപിതചരിത്രവും തയ്യാറായി. അവർ 'പതിത'യാണെന്നു പറഞ്ഞുകൊണ്ടുതന്നെയാണിത്. പതിതയെങ്കിലും നമ്പൂതിരിമാരുടെ പരമ്പരാഗത വൈദികാധികാരത്തിനെതിരെ പ്രവർത്തിച്ചവൾ എന്ന (ഭാഗിക) അംഗീകാരമാണ് ഇവിടെ താത്രിക്കു ലഭിച്ചത്. താത്രി മൂലമുണ്ടായ പൊട്ടിത്തെറി നമ്പൂതിരിസമുദായപരിഷ്ക്കർത്താക്കൾ രംഗത്തെത്തുംമുമ്പ് ഉണ്ടായതാണ്. അതിന്റെ ആഘാതത്തിൽനിന്നാണ് നമ്പൂതിരിപരിഷ്ക്കരണം എന്ന ആശയവും ആവശ്യവും വ്യക്തമായ രൂപം കൈക്കൊണ്ടതെന്ന് വാദിക്കാവുന്നതാണ്. എന്നാൽ (പിൽക്കാലത്തെ) അന്തർജനം സമുദായപരിഷ്ക്കർത്താവായ പുരുഷന്റെ അധികാരത്തെയാണ് നിഷേധിച്ചതെങ്കിൽ അതു മാപ്പാക്കാൻപറ്റാത്ത പാതകമായിപ്പോയേനെ! ഉമാദേവി നരിപ്പറ്റ എന്ന അന്തർജനത്തെപ്പറ്റി സമുദായപരിഷ്ക്കർത്താവായ വി.ടി. ഭട്ടതിരിപ്പാട് - നമ്പൂതിരിസ്ത്രീകൾ പാരമ്പര്യത്തിനെതിരെ തുറന്ന സമരത്തിനുതന്നെ തയ്യാറാകണമെന്ന് വാദിച്ച വ്യക്തി - കർമ്മവിപാകം എന്ന ആത്മകഥാപരമായ കൃതിയിൽ പരാമർശിച്ചത് വായിച്ചാൽ ഇതു വ്യക്തമാകും. (കാണുക \ref{uma-vt}) സമുദായത്തെ പുത്തൻരീതികളിൽ പുനഃക്രമീകരിക്കുകയെന്ന പദ്ധതിക്കനുസരിച്ചു നിൽക്കാത്ത ഉമാദേവി 'കൊള്ളരുത്താത്തവളാ'ണെന്ന വിലയിരുത്തലാണ് വി.ടി.യുടേത്. നമ്പൂതിരിഭർത്താവിനെ ഉപേക്ഷിച്ച് അന്യജാതിയിൽപ്പെട്ടവരെ വിവാഹംകഴിച്ചുവെന്നതാണ് പ്രധാന കുറ്റം - അതും സ്വന്തമിഷ്ടപ്രകാരം, സ്വയം തീരുമാനിച്ച്. കുറിയേടത്തു താത്രിയെ സമുദായനവീകരണചരിത്രത്തിൽ ഉൾപ്പെടുത്തുന്നവർക്കും ഉമാദേവി അനഭിമതയാകുന്നു!


\captionof{mybox}{'കുറിയേടത്തു താത്രി കേസ്'}\label{ch11box2} % place the caption
\begin{tcolorbox}[%
  breakable, % make the box breakable
  arc=0mm, 
  left=1pt, right = 1pt, 
  boxrule=0mm,
  colback = {blue!10}, % since shadow-gray was not defined
] 

\paragraph{}നമ്പൂതിരിസ്ത്രീകൾ വ്യഭിചാരം ചെയ്യുന്നുവെന്ന് സംശയിക്കപ്പെട്ടാൽ, അവരെ സമുദായവിചാരണയ്ക്കു പാത്രമാക്കിയിരുന്നു: 'സ്മാർത്തവിചാരം' എന്നാണ് ഈ സമ്പ്രദായത്തെ വിളിച്ചിരുന്നത്. വ്യഭിചാരക്കുറ്റം ആരോപിക്കപ്പെട്ട സ്ത്രീയെ കുടുംബക്കാരിൽനിന്നും മാറ്റി ഇല്ലത്തിന്റെ മറ്റൊരുഭാഗത്ത് പാർപ്പിച്ച് അവളെ കടുത്ത ചോദ്യംചെയ്യലിനു വിധേയമാക്കിയിരുന്നു. സ്ത്രീ കുറ്റംസമ്മതിച്ചാൽമാത്രമെ വിചാരണ നടത്തുന്നവർക്ക് അവളെ ശിക്ഷിക്കാൻ കഴിയുമായിരുന്നുള്ളൂ; കുറ്റസമ്മതം നടന്നുകഴിഞ്ഞാൽ സ്ത്രീയെ സമുദായത്തിൽനിന്നുതന്നെ പുറത്താക്കിയിരുന്നു.
ഇത്തരമൊരു കേസ് 20-ാം നൂറ്റാണ്ടിന്റെ ആദ്യവർഷങ്ങളിൽ കൊച്ചീരാജ്യത്ത് വലിയ ഒച്ചപ്പാടിനിടവരുത്തി. 1905-'06 കാലത്തായിരുന്നു അത്. 'കുറിയേടത്തു സാവിത്രി' എന്നായിരുന്നു ദോഷം ആരോപിക്കപ്പെട്ട അന്തർജനത്തിന്റെ പേര്. സ്മാർത്തവിചാരം > കാണുക പുറം 34, 35 < നടക്കുമ്പോൾ കുറ്റംസമ്മതിച്ച സ്ത്രീ തന്റെകൂടെ തെറ്റുചെയ്ത പുരുഷന്മാരുടെ പേരുകൾ വെളിപ്പെടുത്തുകയും അവളുടെയൊപ്പം അവരും സമുദായത്തിൽനിന്ന് പുറത്താവുകയും ചെയ്യുന്ന രീതിയായിരുന്നു പതിവ്. എന്നാൽ ഈ കുറിയേടത്തു താത്രി (സാവിത്രി)യുടെ കാര്യത്തിൽ ഒരു വിശേഷമുണ്ടായി. അവർ ഒന്നല്ല, 64 പുരുഷന്മാരുടെ പേരുകൾ പറഞ്ഞു. അവരോടൊപ്പം 64പേരും പുറത്താകേണ്ടിയിരുന്നു. ഇവരിൽ നമ്പൂതിരി, നായർ, അമ്പലവാസി തുടങ്ങി പല സമുദായങ്ങളിലുമുൾപ്പെട്ട പുരുഷന്മാരുണ്ടായിരുന്നു. കൊച്ചീരാജ്യത്തിലെ വലിയ ഇല്ലങ്ങളിലെ പുരുഷന്മാർ പലരും താത്രിയുടെ ലിസ്റ്റിൽ ഉണ്ടായിരുന്നതുകൊണ്ട് ഈ കേസിൽ പുരുഷന്മാരുടെ വശംകൂടി കണക്കാക്കണമെന്ന അഭിപ്രായമുണ്ടായി. കൊച്ചീരാജാവ് അതിന് അനുമതി കൊടുക്കുകയുംചെയ്തു. അന്തർജനത്തെ കൊച്ചിയിൽ കൊണ്ടുവന്നു പാർപ്പിച്ചു. അവിടെ 'പുരുഷവിചാരം' സംഘടിപ്പിച്ചു - അതായത്, താത്രി പേരെടുത്തുപറഞ്ഞ പുരുഷന്മാരെക്കൊണ്ട് അവരെ ചോദ്യംചെയ്യിപ്പിച്ചു. പക്ഷേ, യാതൊരു ഫലവുമുണ്ടായില്ല. തങ്ങൾ കുറിയേടത്തു താത്രിയുമായി സഹവസിച്ചിട്ടില്ലെന്നു തെളിയിക്കാൻ ഇവർക്കാർക്കും കഴിഞ്ഞില്ലത്രെ! താത്രിയോടൊപ്പം അവരെല്ലാം സമുദായത്തിനു പുറത്തായി.
കുറിയേടത്തു താത്രിക്ക് പിന്നീടെന്തു സംഭവിച്ചു എന്ന് നമുക്കറിയില്ല. അവർ കോയമ്പത്തൂരിലേക്കു തീവണ്ടി കയറിയെന്നാണ് പറഞ്ഞുകേൾക്കുന്നത്.
\end{tcolorbox}


\captionof{mybox}{നിവർത്തന പ്രക്ഷോഭം}\label{ch11box3} % place the caption
\begin{tcolorbox}[%
 breakable, % make the box breakable
  arc=0mm, 
  left=1pt, right = 1pt, 
  boxrule=0mm,
  colback = {blue!10}, % since shadow-gray was not defined
] 
\end{tcolorbox}

\captionof{mybox}{വിമോചനസമരത്തിലെ സ്ത്രീപങ്കാളിത്തം}\label{ch11box4} % place the caption
\begin{tcolorbox}[%
 breakable, % make the box breakable
  arc=0mm, 
  left=1pt, right = 1pt, 
  boxrule=0mm,
  colback = {blue!10}, % since shadow-gray was not defined
] 
\end{tcolorbox}

\captionof{mybox}{മാപ്പിളകലാപം : സ്ത്രീകളുടെ ഓർമ്മകൾ}\label{ch11box5} % place the caption
\begin{tcolorbox}[%
 breakable, % make the box breakable
  arc=0mm, 
  left=1pt, right = 1pt, 
  boxrule=0mm,
  colback = {blue!10}, % since shadow-gray was not defined
] 
\end{tcolorbox}

\section{ദേശീയസമരത്തിന്റെ വേലിയേറ്റത്തിൽ സ്ത്രീകൾ}
\label{11.3}

\captionof{mybox}{ലക്ഷ്മി എൻ മേനോൻ (1897- 1994)}\label{ch11box6} % place the caption
\begin{tcolorbox}[%
 breakable, % make the box breakable
  arc=0mm, 
  left=1pt, right = 1pt, 
  boxrule=0mm,
  colback = {blue!10}, % since shadow-gray was not defined
] 
\end{tcolorbox}


\captionof{mybox}{എ.വി. കുട്ടിമാളു അമ്മ}\label{ch11box7} % place the caption
\begin{tcolorbox}[%
 breakable, % make the box breakable
  arc=0mm, 
  left=1pt, right = 1pt, 
  boxrule=0mm,
  colback = {blue!10}, % since shadow-gray was not defined
] 
\end{tcolorbox}
\section{ഇടതുപക്ഷത്തും തൊഴിലാളിസമരങ്ങളിലും സ്ത്രീകൾ}
\label{11.4}
\captionof{mybox}{രാഷ്ട്രീയമുണ്ടെങ്കിൽ കുടുംബജീവിതമില്ല?}\label{ch11box13} % place the caption
\begin{tcolorbox}[%
 breakable, % make the box breakable
  arc=0mm, 
  left=1pt, right = 1pt, 
  boxrule=0mm,
  colback = {blue!10}, % since shadow-gray was not defined
] 

ആലപ്പുഴയിലെ കമ്യൂണിസ്റ്റ് കലാപ്രവർത്തകയായ പി.കെ. മേദിനി തന്റെ വിവാഹത്തെക്കുറിച്ചു പറഞ്ഞത്:

\begin{quotation}


....രാത്രികാലങ്ങളിൽ പാട്ടുപാടി ചെറിയൊരു വരുമാനമൊക്കെ കിട്ടിത്തുടങ്ങി. സഹോദരന്മാരൊക്കെ വിവാഹം കഴിച്ചു. സ്ത്രീധനം കൊടുത്തു കല്യാണംകഴിക്കാൻ എനിക്കു നിവൃത്തിയുണ്ടായിരുന്നില്ല. എന്റെ വിവാഹത്തിന്റെ ധനശേഖരണാർത്ഥം 'നിങ്ങളെന്നെ കമ്യൂണിസ്റ്റാക്കി' എന്ന നാടകം നടത്താൻ എന്റെ സഹോദരന്മാർ പ്ലാനിട്ടു. അതെനിക്കു വലിയ വിഷമമുണ്ടാക്കി. എനിക്ക് 21 വയസ്സു പ്രായം. ഇതിനിടെ ഒരുപാടു പ്രേമങ്ങൾ വന്നുകൊണ്ടിരുന്നു. ഒന്നിലും നല്ലജീവിതം കിട്ടില്ലെന്നുറപ്പായപ്പോൾ ഞാനവ തിരസ്ക്കരിച്ചു. പിന്നെ ഒരു തീരുമാനമെടുത്തു. ഒരു കോൺഗ്രസ്സുകാരനായ എന്റെ കസിൻ, ഈ വീടിന്റെ ഉടമ, എന്റെ ആങ്ങളമാരുടെ സുഹൃത്താണ്, വീട്ടിലെപ്പോഴും വരും. വിദ്യാഭ്യാസമുള്ള ആളാണ്. സുമുഖനും സുന്ദരനുമാണ്. പക്ഷേ, ചില ആരോഗ്യപ്രശ്നങ്ങളുണ്ട് - ആസ്ത്മ. അതിനാൽ വിവാഹംകഴിക്കാതെ നടക്കുകയാണ്. എനിക്ക് ചേട്ടനെ വലിയ ഇഷ്ടമായിരുന്നു. നാടകം നടത്തി കല്യാണച്ചെലവുണ്ടാക്കാൻ ശ്രമിക്കുന്ന ആങ്ങളമാരേക്കാൾ എത്ര ഉയർന്ന മനുഷ്യനാണദ്ദേഹം. അദ്ദേഹത്തോടു ഞാൻ പ്രേമാഭ്യർത്ഥന നടത്തി...
\flushright{(അഭിമുഖം, പച്ചക്കുതിര, ജൂൺ 2008, പുറം. 36)}
\end{quotation}
\end{tcolorbox}
