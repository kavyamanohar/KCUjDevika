\chapter{സ്ത്രീശരീരത്തിനു ചുറ്റും നടന്ന യുദ്ധങ്ങൾ!}
\label{chapter7}
\section{ശരീരത്തെ കൊണ്ടുനടക്കേണ്ട ഭാരം}
\label{ch7sec1}
\section{മാറുമറയ്ക്കലും നാണവും മാനവും}
\label{ch7sec2}
\section{ശരീരത്തെ കൊണ്ടുനടക്കേണ്ട ഭാരം}
\label{ch7sec3}
\section{സ്ത്രീകളുടെ വസ്ത്രധാരണവും സാമൂഹികപരിഷ്ക്കരണവും}
\label{ch7sec4}
\captionof{mybox}{നമ്പൂതിരിസമുദായപരിഷ്കരണ പ്രസ്ഥാനത്തിന്റെ നായികമാർ}\label{ch7box2} % place the caption
\begin{tcolorbox}[%
 breakable, % make the box breakable
  arc=0mm, 
  left=1pt, right = 1pt, 
  boxrule=0mm,
  colback = {blue!10}, % since shadow-gray was not defined
] 
\noindent 1930കളിൽ നമ്പൂതിരിസമുദായപരിഷ്കരണപ്രസ്ഥാനത്തിന്റെ മുൻനിരയിലേക്ക് കഴിവുറ്റ സ്ത്രീകൾ വരാൻതുടങ്ങി. മഹിള എന്ന സ്ത്രീപ്രസിദ്ധീകരണം 'നമ്പൂതിരിസാമ്രാജ്യ'ത്തിന്റെ [വീരനായികയായ] 'ജോവാൻ ഓഫ് ആർക്ക്' എന്നു വിശേഷിപ്പിച്ച പാർവ്വതി നെന്മിനിമംഗലമായിരുന്നു അവരിൽ പ്രധാനി. ആധുനികവസ്ത്രംധരിച്ച് യോഗക്ഷേമസഭാസമ്മേളനത്തിൽ ആദ്യമായി ഒരു സ്ത്രീ - പാർവ്വതി മനേഴി - പങ്കെടുത്തതിനെത്തുടർന്ന് സമ്മേളനത്തിൽ പങ്കെടുത്ത സ്ത്രീകളുടെ എണ്ണം വർദ്ധിച്ചുകൊണ്ടേയിരുന്നു. ഇടക്കുന്നിയിൽ നടന്ന യോഗക്ഷേമസഭയുടെ 22-ാം സമ്മേളനത്തിൽ ഒരൊറ്റ സ്ത്രീമാത്രമേ ഉണ്ടായിരുന്നുള്ളു. എന്നാൽ കാറൽമണ്ണവെച്ചു കൂടിയ 25-ാം സമ്മേളനത്തിൽ 75 അന്തർജനങ്ങൾ സന്നിഹിതരായിരുന്നു. നമ്പൂതിരിയുവജനസംഘത്തിലൂടെയാണ് ഇവർ വളർന്നത്. 1930കളിൽ ആര്യാപള്ളം, പാർവ്വതി നെന്മിനിമംഗലം, ദേവകി നരിക്കാട്ടിരി തുടങ്ങിയവർ അന്തർജനങ്ങൾക്കു സഹിക്കേണ്ടിവന്നിരുന്ന നിയന്ത്രണങ്ങളെ പരസ്യമായി തള്ളിക്കളഞ്ഞുകൊണ്ട് പൊതുപ്രവർത്തകരായി മാറി. ആധുനികരീതിയിൽ വസ്ത്രധാരണംചെയ്യാനും യാത്രചെയ്യാനും പൊതുവേദികളിൽ പ്രസംഗിക്കാനും നിയമസഭകളിൽ പ്രവർത്തിക്കാനും പ്രാപ്തിയുള്ള അന്തർജനങ്ങൾ ഇക്കാലത്ത് പൊതുരംഗത്തേക്കു കടന്നു. 1940കളിലെ രാഷ്ട്രീയസമരങ്ങളിൽ - പാലിയംസത്യാഗ്രഹത്തിലും കമ്യൂണിസ്റ്റുപാർട്ടിയുടെ പ്രവർത്തനത്തിലും മറ്റും - അവർ പങ്കുചേരാൻ തുടങ്ങി. ലളിതാംബിക അന്തർജനം അക്കാലമായപ്പോഴേക്കും ആധുനികകേരളീയസാഹിത്യരംഗത്ത് തന്റേതായ ഇടമുണ്ടാക്കിക്കഴിഞ്ഞിരുന്നു.
\end{tcolorbox}

